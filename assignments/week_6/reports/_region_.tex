\message{ !name(Report.tex)}\documentclass[]{article}
\usepackage{csquotes}
\usepackage{graphicx}
\usepackage{amsmath}
\usepackage{amssymb}
\usepackage{bm}
\usepackage{tikz}
\usetikzlibrary{shapes, arrows}
\graphicspath{ {../figs/} }


% Title Page
\title{Industrial revolution 2.0: using variational autoencoders to produce images of clothing}
\author{Adriaan Hilbers}


\begin{document}

\message{ !name(Report.tex) !offset(57) }
$q_{\phi}(\bm{y}^{(i)}|\bm{x}^{(i)})$
\end{itemize}


This section introduces the setup of the model and how it is trained. For sake of brevity, only a concise overview is provided. The setup of the model is shown in Figure \ref{fig:neuralnet}. Overall, a feedforward run of the variational autoencoder works as follows, as inspired by \cite{notes, frans, altosaar, shafkat}: 
\begin{itemize}
\item The raw input is an image encoded as a length 784 vector with the pixel brightnesses. 
\item This image is reshaped and fed through a convolutional neural network consisting of both convolutional and dense layers. This gives, for each image, two vectors: a mean
\message{ !name(Report.tex) !offset(167) }

\end{document}          
